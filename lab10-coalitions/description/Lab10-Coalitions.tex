\documentclass[a4paper, 12pt, twoside]{article}
\newcommand{\languagesetting}{english}
\newcommand{\geometrysetting}{margin=2cm}
\newcommand{\elementsroot}{../../}
\usepackage{\elementsroot/quick}
\usepackage{\elementsroot/sendbox}
\newcommand{\noulem}{}\renewcommand{\picLocal}{}

\newcommand{\makepageTwo}{\newpage\label{res}}\newcommand{\pageTwo}{page \pageref{res}}

\begin{document}
% TODO
\spk{-3ex}\dotitle[26.05.2021]{MAS: Activity 10 \textendash\ Agent Coalitions}{Alexandru Sorici}
\spk{1.5ex}\vskip-3ex

In this activity, the objective is to implement mechanisms for forming \textit{stable} agent coalitions, whereby the purpose of forming the coalition is that of obtaining a higher utility, than is possible for each individual agent (given its resources).

The problem setting is that of a series of \textit{buyer} agents, each with constrained resources (money). There agents must obtain one instance (or part thereof) \textit{from each type} of a set or products. For each product type, there are several available versions, each with a value and a cost.

The agent resource are given in such a way, that each agent cannot single-handedly acquire instances of each product type. Therefore, the agents have to form coalitions, where they \textit{cooperate} in order to buy higher-valued product items.
%
\bigskip
%
While doing so, two aspects have to be observed:
%
\begin{itemize}
	\item The formed coalition of agents has to be \emph{stable}, i.e. no agent must have the \textit{rational} incentive to leave this coalition for another.
	\item To maintain stability, \emph{fairness} of the value distribution among coalition agents has to be ensured. This means that agents have to receive a total value that is \textit{proportional} to their \textit{contribution} to the coalition.
\end{itemize}
%
\bigskip
%
\textbf{Agent Setup.}

The problem setup is similar to the Ice-Cream Game from slide 20 of the attached class presentation. However, in our setup there are two product types (\texttt{r1} and \texttt{r2}), each with a given lineup of product instances, characterized by value and price.
%
The agents must consider the \textit{fairness} distribution for each product type (i.e. a value distribution proportional, \textit{for that product type}, with the contribution \textit{for that product type}).

The \textbf{value} of a coalition is 0 if it cannot buy both types of products. If it can buy both types of products, the value of the coalition is given by the sum of values of the product instances.

The task of the agents is to interact with one another and propose that they form a coalition. 
To simplify the task of carrying out coalition formation negotiations, the agents are given turns in which they opt to \textit{join} a new coalition or to \textit{disband} their existing one. \textbf{At any moment, an agent can be part of a single coalition.}


Each coalition has a \textit{coalition leader}, which is the agent that negotiates on behalf of the coalition. The \textit{coalition leader} is the agent which has \textit{the highest amount of resources (money)} in the coalition. If two agents have the same amount of resources, they are sorted in alphabetical order of their names.

%
\bigskip
%
\textbf{Communication Setup.}
The \texttt{Environment} manages the following interactions: 

\begin{itemize}
	\item It handles a token to the \textit{buyer} agents in \textit{decreasing} order of their resources (money). Only the agent which has a token is allowed to carry out actions. 
	%
	\item The actions available to an agent are: (i)\textbf{request} to join a coalition, (ii)\textbf{disband} a coalition \textbf{of which it is a member}, (iii)\textbf{accept the request}, made by another agent, to join the coalition, (iv)\textbf{deny the request}, made by another agent, to join a coalition (by also providing a counter offer product share)
	%
	\item A \textbf{request to join a coalition} is made by specifying a \textit{product\_share}: the \textit{amount of money} the agent wants to contribute to \textit{each type} of resource; the \textit{value} the agent wants to obtain from each product type
	%
	\item At each turn buyer agent:
		\begin{itemize}
			\item receive a list of \texttt{CoalitionActions} that include: \textbf{requests} from other agents to join the coalition of the current buyer agent, with a given product share offer; \textbf{denials} to join a coalition and a counter offer for the product share
			%
			\item submit a (possibly empty) list of actions to be carried out by the environment, from the set mentioned previously
		\end{itemize}
\end{itemize}

The simulation ends when there is no change in coalition formations, from one token pass to the other, or when \textbf{MAX\_ROUNDS} of token passes have elapsed.

% \sendbox{alex.sorici+mas@gmail.com}
% \sendbox[a]{cs@andreiolaru.ro}
\end{document}
