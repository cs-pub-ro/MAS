\documentclass[a4paper, 12pt, twoside]{article}
\newcommand{\languagesetting}{english}

\newcommand{\elementsroot}{../../}
\usepackage{\elementsroot/quick}
\usepackage{\elementsroot/sendbox}


\newcommand{\makepageTwo}{\newpage\label{res}}
\newcommand{\pageTwo}{page \pageref{res}}

% TODO year + ID
%\newcommand{\reslink}{\rf{http://cs.curs.pub.ro/2015/mod/folder/view.php?id=5498}}

\begin{document}
% TODO
\spk{-2ex}\dotitle[12.05.2021]{MAS: Activity 9 \textendash\ Auctions and Negotiations}{Alexandru Sorici}
\spk{2ex}5\vskip-5ex

The Multi-Agent Oriented Programming (MAOP) Paradigm is well suited for application scenarios that involve interactions between multiple parties that have a stake in a given objective, though possibly different interests.

This assignment aims to make use of two interactions that are typically encountered in the multi-agent literature: auctions and negotiations.

\section{Problem Description}
The problem description that makes use of these interactions is that of \textit{a company (ACME) that wants to build a new headquarters and needs to sign contracts for the following items: (i) structural design, (ii) building, (iii) electrical and plumbing work, (iv) interior design.}

The contracting occurs by the following assumptions and rules:
\begin{itemize}
	\item The contracting company is called ACME and it \textbf{has a given budget each construction item};
	\item The contractors are companies A -- F. Each company is specialized in a subset of the 4 construction tasks (e.g. structural design and interior design, building and electrical/plumbing work, etc). Each company has a \textbf{lower limit for the price at which it will take on each type of construction task} -- its actual cost.
	
	\item The goal of ACME is to \textbf{complete the headquarters} and to \textbf{save as much money as possible} from the overall budget.
	
	\item The goal of each contractor company is to participate in \textbf{at least one contract}, since this builds up serious reputation. At the same time, each company wants to ensure it turns in as much of a profit as possible.
\end{itemize}

To perform the contracting for each task, ACME employs the following strategy:
\begin{itemize}
	\item It first holds a variation of the \textbf{Dutch (descending auction)} (\textbf{see Negotiation lecture}) to establish a preliminary list of companies who are willing to do the task at a given price. The variation is that ACME will start from a low price and continuously raise it, until one (or more) companies place a bid. ACME is allowed to raise the price \textbf{at most 3 times}. 

	\item With each interested company, ACME enters a negotiation using the \textbf{monotonic concession protocol} (see Negotiation lecture) to try and bring the price further down. ACME holds at most 3 rounds of concession. After the negotiations, it will select the company with the lowest price (or one randomly if each has the same offer).
\end{itemize}

\section{Multi-Agent Modeling}
%
The game setup follows these indications:
\begin{itemize}
	\item Model ACME and each of the contractor companies as an agent.
	\item The environment keeps a record of each agent and knows what services they can provide (i.e. structural design, building, electrical and plumbing work, interior design).
	%	
	\item The environment maintains a state of two different interaction phases: \textit{auction phase} and \textit{negotiation phase}. 
	%
	\item In the \textit{auction phase} the environment keeps track of current auction item, current auction round, companies that have secured a spot to the negotiation phase
	%
	\item In the \textit{negotiation phase} the environment will facilitate an \texttt{MonotonicConcessionNegotiation} instance for each of the agents that were selected by ACME during the \textit{auction phase}.
		\begin{itemize}
			\item ACME is always the initiator of the negotiation. ACME will try to start at lower price, and bring it back up to the value of auction phase.
			\item Company agents are always the responders in the negotiation protocol. Company agents decide if they accept lowering the received budget for the construction item.
			\item Company agents \textbf{do not know} the offers made in the negotiation by other agents, but they \textbf{do know} how many agents have entered the negotiation.
		\end{itemize}
	
\end{itemize}


\section{Other specifications}
Each student has to implement a strategy for ACME and a strategy used by company agents.
Final lab evaluation will be \textbf{team based} one: two solutions will be pitted against each other - one student gives the strategy for ACME, while the other one gives the strategy for the company agents.
The evaluation file will be based on a config similar to test setup number 2 (see below).


For testing purposes, you have two test setups. In both setups, ACME has the following budget:
\begin{itemize}\footnotesize
	\item structural design: 5000
	\item building: 10000
	\item electric/plumbing: 4000
	\item interior design: 5000
\end{itemize}

For the contractors:
\begin{itemize}
	\item In the first one (\texttt{config-companies-1.cfg}) you have a manual setup of the company cost values for each contracting stage.
		\begin{itemize}
			\item company A: structural design - 5000, interior design: 5000
			\item company B: structural design - 4000, building - 8000
			\item company C: building - 9000, interior design: 3500
			\item company D: building - 7500, electric and plumbing - 2500
			\item company E: building - 7200, electric and plumbing - 3700
			\item company F: structural design - 4000, interior design: 4000
		\end{itemize}
	%
	\item In the second one (\texttt{config-companies-2.cfg}), the cost values are distributed to companies according to a \emph{normal distribution}, with mean equal to the budget of ACME and a standard deviation of 20\% from the max budget (e.g. for building, mean = 10000, std = 2000). For this second setup, \textit{both ACME and the contractor companies know how budget and costs are distributed!}
	
\end{itemize}

\end{document}
